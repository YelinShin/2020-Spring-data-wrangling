% Options for packages loaded elsewhere
\PassOptionsToPackage{unicode}{hyperref}
\PassOptionsToPackage{hyphens}{url}
%
\documentclass[
  11pt,
]{article}
\usepackage{lmodern}
\usepackage{amssymb,amsmath}
\usepackage{ifxetex,ifluatex}
\ifnum 0\ifxetex 1\fi\ifluatex 1\fi=0 % if pdftex
  \usepackage[T1]{fontenc}
  \usepackage[utf8]{inputenc}
  \usepackage{textcomp} % provide euro and other symbols
\else % if luatex or xetex
  \usepackage{unicode-math}
  \defaultfontfeatures{Scale=MatchLowercase}
  \defaultfontfeatures[\rmfamily]{Ligatures=TeX,Scale=1}
\fi
% Use upquote if available, for straight quotes in verbatim environments
\IfFileExists{upquote.sty}{\usepackage{upquote}}{}
\IfFileExists{microtype.sty}{% use microtype if available
  \usepackage[]{microtype}
  \UseMicrotypeSet[protrusion]{basicmath} % disable protrusion for tt fonts
}{}
\makeatletter
\@ifundefined{KOMAClassName}{% if non-KOMA class
  \IfFileExists{parskip.sty}{%
    \usepackage{parskip}
  }{% else
    \setlength{\parindent}{0pt}
    \setlength{\parskip}{6pt plus 2pt minus 1pt}}
}{% if KOMA class
  \KOMAoptions{parskip=half}}
\makeatother
\usepackage{xcolor}
\IfFileExists{xurl.sty}{\usepackage{xurl}}{} % add URL line breaks if available
\IfFileExists{bookmark.sty}{\usepackage{bookmark}}{\usepackage{hyperref}}
\hypersetup{
  pdftitle={Corona virus data wrangling},
  pdfauthor={Yelin Shin},
  hidelinks,
  pdfcreator={LaTeX via pandoc}}
\urlstyle{same} % disable monospaced font for URLs
\usepackage[margin=1in]{geometry}
\usepackage{longtable,booktabs}
% Correct order of tables after \paragraph or \subparagraph
\usepackage{etoolbox}
\makeatletter
\patchcmd\longtable{\par}{\if@noskipsec\mbox{}\fi\par}{}{}
\makeatother
% Allow footnotes in longtable head/foot
\IfFileExists{footnotehyper.sty}{\usepackage{footnotehyper}}{\usepackage{footnote}}
\makesavenoteenv{longtable}
\usepackage{graphicx,grffile}
\makeatletter
\def\maxwidth{\ifdim\Gin@nat@width>\linewidth\linewidth\else\Gin@nat@width\fi}
\def\maxheight{\ifdim\Gin@nat@height>\textheight\textheight\else\Gin@nat@height\fi}
\makeatother
% Scale images if necessary, so that they will not overflow the page
% margins by default, and it is still possible to overwrite the defaults
% using explicit options in \includegraphics[width, height, ...]{}
\setkeys{Gin}{width=\maxwidth,height=\maxheight,keepaspectratio}
% Set default figure placement to htbp
\makeatletter
\def\fps@figure{htbp}
\makeatother
\setlength{\emergencystretch}{3em} % prevent overfull lines
\providecommand{\tightlist}{%
  \setlength{\itemsep}{0pt}\setlength{\parskip}{0pt}}
\setcounter{secnumdepth}{-\maxdimen} % remove section numbering
\usepackage{booktabs}
\usepackage{longtable}
\usepackage{array}
\usepackage{multirow}
\usepackage{wrapfig}
\usepackage{float}
\usepackage{colortbl}
\usepackage{pdflscape}
\usepackage{tabu}
\usepackage{threeparttable}
\usepackage{threeparttablex}
\usepackage[normalem]{ulem}
\usepackage{makecell}
\usepackage{xcolor}

\title{Corona virus data wrangling}
\author{Yelin Shin}
\date{5/4/2020}

\begin{document}
\maketitle

\hypertarget{github-repository}{%
\section{Github repository:}\label{github-repository}}

\hypertarget{httpsgithub.comyelinshin2020-spring-data-wrangling}{%
\subsubsection{\texorpdfstring{\url{https://github.com/YelinShin/2020-Spring-data-wrangling}}{https://github.com/YelinShin/2020-Spring-data-wrangling}}\label{httpsgithub.comyelinshin2020-spring-data-wrangling}}

\hypertarget{introduction}{%
\section{Introduction:}\label{introduction}}

Around late December 2019, people got notice about a virus that was
speared from China. After the virus got spread quickly, WHO named it as
Covid-19 and decided as epidemic. However, at that time, people in
non-Asian country underestimated how fast the virus can be spread to
their countries. Therefore, most of the countries were not prepared and
faced rapid increase of confirmed cases and even death cases from the
virus. Also, it is relative to our class because of the virus, every
classes hold lectures remotely. Therefore, in current situation, people
wonder when the virus have a lull. I decided to show the changes by
various data visualization.

\hypertarget{data-set-1}{%
\section{Data set (1):}\label{data-set-1}}

\begin{enumerate}
\def\labelenumi{\arabic{enumi}.}
\item
  The main world time series dataset is from GitHub,
  ``\url{https://github.com/datasets/covid-19/blob/master/data/time-series-19-covid-combined.csv}''.
  Since the data contains detail information about the province/state in
  each country, I summarize each cases number by country and date. So,
  it will contain only one pair of (country, date).
\item
  The time series corona case file contains 2018 population of each
  countries. Therefore, I decided to change the number to 2019
  population from Wikipedia,
  \url{https://en.wikipedia.org/wiki/List_of_countries_by_population_(United_Nations)}
  by web-scrapping the page. Since the population formatted with comma,
  I erased the comma then converted it to number type
\item
  Also, I grab the time-series update of the number of tested cases in
  each country by
  ``\url{https://github.com/owid/covid-19-data/blob/master/public/data/testing/covid-testing-all-observations.csv}''.
  This csv file contains too many information and the country name
  contains `- tests performed', I extract the country name only and the
  current update of testing number. Moreover, some countries have 2
  resources to track the testing number.
\item
  So, I fix one resource per country. After I got the 3 cleaned data, I
  use join to get the finalized data that contains all the cases, test,
  and population. Moreover, I make a table for latest updated case
  number for each country. Since some countries did not update/share the
  testing number every day, I grab most recent number of testing into
  this table, and put active number by mutate. (Active = confirmed --
  deaths -- recovered). When I joined the latest Covid and latest
  testing data, I based on latest Covid date since sometime the
  publisher update testing dataset faster than Covid dataset
\item
  Lastly, I get the world ranking by GDP from Wikipedia
  (``\url{https://en.wikipedia.org/wiki/List_of_countries_by_GDP_(nominal)}'').
  Since there are too many countries in the data, it is better to show
  subset of countries. Therefore, it is good to show the economically
  developed countries for data visualization.
\end{enumerate}

For all data source, I edit the countries' name to match with
`country.regions' to use chroplethrMaps.

~

\hypertarget{raw-table-looks-like}{%
\subsection{Raw table looks like \ldots{}}\label{raw-table-looks-like}}

\begin{longtable}[]{@{}lllrrrrr@{}}
\toprule
Date & Country/Region & Province/State & Lat & Long & Confirmed &
Recovered & Deaths\tabularnewline
\midrule
\endhead
2020-02-11 & Monaco & NA & 43.7333 & 7.4167 & 0 & 0 & 0\tabularnewline
2020-04-16 & Eswatini & NA & -26.5225 & 31.4659 & 16 & 8 &
1\tabularnewline
\bottomrule
\end{longtable}

\textbf{Note:} \textsuperscript{a} Random 2 rows in world time-series
covid 19

\begin{longtable}[]{@{}lllllrrrrrrrr@{}}
\toprule
\begin{minipage}[b]{0.04\columnwidth}\raggedright
Entity\strut
\end{minipage} & \begin{minipage}[b]{0.02\columnwidth}\raggedright
Date\strut
\end{minipage} & \begin{minipage}[b]{0.11\columnwidth}\raggedright
Source URL\strut
\end{minipage} & \begin{minipage}[b]{0.06\columnwidth}\raggedright
Source label\strut
\end{minipage} & \begin{minipage}[b]{0.06\columnwidth}\raggedright
Notes\strut
\end{minipage} & \begin{minipage}[b]{0.02\columnwidth}\raggedleft
Cumulative total\strut
\end{minipage} & \begin{minipage}[b]{0.05\columnwidth}\raggedleft
Daily change in cumulative total\strut
\end{minipage} & \begin{minipage}[b]{0.04\columnwidth}\raggedleft
Cumulative total per thousand\strut
\end{minipage} & \begin{minipage}[b]{0.06\columnwidth}\raggedleft
Daily change in cumulative total per thousand\strut
\end{minipage} & \begin{minipage}[b]{0.04\columnwidth}\raggedleft
3-day rolling mean daily change\strut
\end{minipage} & \begin{minipage}[b]{0.06\columnwidth}\raggedleft
3-day rolling mean daily change per thousand\strut
\end{minipage} & \begin{minipage}[b]{0.04\columnwidth}\raggedleft
7-day rolling mean daily change\strut
\end{minipage} & \begin{minipage}[b]{0.06\columnwidth}\raggedleft
7-day rolling mean daily change per thousand\strut
\end{minipage}\tabularnewline
\midrule
\endhead
\begin{minipage}[t]{0.04\columnwidth}\raggedright
Paraguay - samples tested\strut
\end{minipage} & \begin{minipage}[t]{0.02\columnwidth}\raggedright
2020-04-15\strut
\end{minipage} & \begin{minipage}[t]{0.11\columnwidth}\raggedright
\url{https://github.com/torresmateo/covidpy-rest/blob/master/data/covidpy.csv}\strut
\end{minipage} & \begin{minipage}[t]{0.06\columnwidth}\raggedright
Ministry of Public Health and Social Welfare\strut
\end{minipage} & \begin{minipage}[t]{0.06\columnwidth}\raggedright
Made available on Github by Mateo Torres\strut
\end{minipage} & \begin{minipage}[t]{0.02\columnwidth}\raggedleft
4252\strut
\end{minipage} & \begin{minipage}[t]{0.05\columnwidth}\raggedleft
379\strut
\end{minipage} & \begin{minipage}[t]{0.04\columnwidth}\raggedleft
0.596\strut
\end{minipage} & \begin{minipage}[t]{0.06\columnwidth}\raggedleft
0.053\strut
\end{minipage} & \begin{minipage}[t]{0.04\columnwidth}\raggedleft
291.000\strut
\end{minipage} & \begin{minipage}[t]{0.06\columnwidth}\raggedleft
0.041\strut
\end{minipage} & \begin{minipage}[t]{0.04\columnwidth}\raggedleft
257.857\strut
\end{minipage} & \begin{minipage}[t]{0.06\columnwidth}\raggedleft
0.036\strut
\end{minipage}\tabularnewline
\begin{minipage}[t]{0.04\columnwidth}\raggedright
Serbia - people tested\strut
\end{minipage} & \begin{minipage}[t]{0.02\columnwidth}\raggedright
2020-04-19\strut
\end{minipage} & \begin{minipage}[t]{0.11\columnwidth}\raggedright
\url{https://github.com/aleksandar-jovicic/COVID19-Serbia/blob/master/timeseries.csv}\strut
\end{minipage} & \begin{minipage}[t]{0.06\columnwidth}\raggedright
Ministry of Health\strut
\end{minipage} & \begin{minipage}[t]{0.06\columnwidth}\raggedright
Made available by Aleksandar Jovičić on Github\strut
\end{minipage} & \begin{minipage}[t]{0.02\columnwidth}\raggedleft
38701\strut
\end{minipage} & \begin{minipage}[t]{0.05\columnwidth}\raggedleft
2673\strut
\end{minipage} & \begin{minipage}[t]{0.04\columnwidth}\raggedleft
5.687\strut
\end{minipage} & \begin{minipage}[t]{0.06\columnwidth}\raggedleft
0.393\strut
\end{minipage} & \begin{minipage}[t]{0.04\columnwidth}\raggedleft
3076.333\strut
\end{minipage} & \begin{minipage}[t]{0.06\columnwidth}\raggedleft
0.452\strut
\end{minipage} & \begin{minipage}[t]{0.04\columnwidth}\raggedleft
2912.714\strut
\end{minipage} & \begin{minipage}[t]{0.06\columnwidth}\raggedleft
0.428\strut
\end{minipage}\tabularnewline
\bottomrule
\end{longtable}

\textbf{Note:} \textsuperscript{a} Random 2 rows in world time-series
testing update

\begin{longtable}[]{@{}lllllll@{}}
\toprule
\begin{minipage}[b]{0.03\columnwidth}\raggedright
\strut
\end{minipage} & \begin{minipage}[b]{0.11\columnwidth}\raggedright
Country or area\strut
\end{minipage} & \begin{minipage}[b]{0.16\columnwidth}\raggedright
UN continentalregion{[}4{]}\strut
\end{minipage} & \begin{minipage}[b]{0.16\columnwidth}\raggedright
UN statisticalregion{[}4{]}\strut
\end{minipage} & \begin{minipage}[b]{0.16\columnwidth}\raggedright
Population(1 July 2018)\strut
\end{minipage} & \begin{minipage}[b]{0.16\columnwidth}\raggedright
Population(1 July 2019)\strut
\end{minipage} & \begin{minipage}[b]{0.05\columnwidth}\raggedright
Change\strut
\end{minipage}\tabularnewline
\midrule
\endhead
\begin{minipage}[t]{0.03\columnwidth}\raggedright
179\strut
\end{minipage} & \begin{minipage}[t]{0.11\columnwidth}\raggedright
Martinique\strut
\end{minipage} & \begin{minipage}[t]{0.16\columnwidth}\raggedright
Americas\strut
\end{minipage} & \begin{minipage}[t]{0.16\columnwidth}\raggedright
Caribbean\strut
\end{minipage} & \begin{minipage}[t]{0.16\columnwidth}\raggedright
375,673\strut
\end{minipage} & \begin{minipage}[t]{0.16\columnwidth}\raggedright
375,554\strut
\end{minipage} & \begin{minipage}[t]{0.05\columnwidth}\raggedright
−0.03\%\strut
\end{minipage}\tabularnewline
\begin{minipage}[t]{0.03\columnwidth}\raggedright
23\strut
\end{minipage} & \begin{minipage}[t]{0.11\columnwidth}\raggedright
Italy\strut
\end{minipage} & \begin{minipage}[t]{0.16\columnwidth}\raggedright
Europe\strut
\end{minipage} & \begin{minipage}[t]{0.16\columnwidth}\raggedright
Southern Europe\strut
\end{minipage} & \begin{minipage}[t]{0.16\columnwidth}\raggedright
60,627,291\strut
\end{minipage} & \begin{minipage}[t]{0.16\columnwidth}\raggedright
60,550,075\strut
\end{minipage} & \begin{minipage}[t]{0.05\columnwidth}\raggedright
−0.13\%\strut
\end{minipage}\tabularnewline
\bottomrule
\end{longtable}

\textbf{Note:} \textsuperscript{a} Random 2 rows in world 2019
population

\begin{longtable}[]{@{}llll@{}}
\toprule
& Rank & Country/Territory & GDP(US\$million)\tabularnewline
\midrule
\endhead
120 & 115 & Korea, North & 17,364\tabularnewline
144 & --- & New Caledonia & 9,894\tabularnewline
\bottomrule
\end{longtable}

\textbf{Note:} \textsuperscript{a} Random 2 rows in GDP Ranking

\hypertarget{after-join-and-clean-up}{%
\subsection{After Join and clean
up\ldots{}}\label{after-join-and-clean-up}}

\begin{longtable}[]{@{}llrrrrrr@{}}
\toprule
Date & region & confirmed & recovered & deaths & actives &
cumulative\_test & population\tabularnewline
\midrule
\endhead
2020-04-19 & colombia & 3792 & 711 & 179 & 2902 & 62674 &
50339443\tabularnewline
2020-03-17 & peru & 117 & 1 & 0 & 116 & 2797 & 32510453\tabularnewline
2020-04-23 & finland & 4284 & 2000 & 172 & 2112 & 78464 &
5532156\tabularnewline
2020-02-24 & south africa & 0 & 0 & 0 & 0 & 116 &
58558270\tabularnewline
2020-04-09 & taiwan & 380 & 67 & 5 & 308 & 43886 &
23773876\tabularnewline
\bottomrule
\end{longtable}

\textbf{Note:} \textsuperscript{a} Random 5 row in time-series covid 19
table

\begin{longtable}[]{@{}llrrrrrr@{}}
\toprule
Date & region & confirmed & recovered & deaths & actives & population &
cumulative\_test\tabularnewline
\midrule
\endhead
2020-05-03 & lithuania & 1410 & 635 & 46 & 729 & 2759627 &
141678\tabularnewline
2020-05-03 & latvia & 879 & 348 & 16 & 515 & 1906743 &
64245\tabularnewline
2020-05-03 & united kingdom & 187842 & 901 & 28520 & 158421 & 67530172 &
1206405\tabularnewline
2020-05-03 & qatar & 15551 & 1664 & 12 & 13875 & 2832067 &
104435\tabularnewline
2020-05-03 & uruguay & 655 & 442 & 17 & 196 & 3461734 &
21164\tabularnewline
\bottomrule
\end{longtable}

\textbf{Note:} \textsuperscript{a} Random 5 row in most recent covid 19
table

\begin{longtable}[]{@{}lr@{}}
\toprule
region & rank\tabularnewline
\midrule
\endhead
united states of america & 1\tabularnewline
china & 2\tabularnewline
japan & 3\tabularnewline
germany & 4\tabularnewline
united kingdom & 5\tabularnewline
france & 6\tabularnewline
india & 7\tabularnewline
brazil & 8\tabularnewline
italy & 9\tabularnewline
canada & 10\tabularnewline
\bottomrule
\end{longtable}

\textbf{Note:} \textsuperscript{a} World ranking top 10 by GDP ~ \#
Further data visualization

\hypertarget{most-updated-number-of-cases-time-series-for-cases}{%
\subsection{Most updated number of cases \& time-series for
cases}\label{most-updated-number-of-cases-time-series-for-cases}}

Now we can get the most updated number of each cases (confirmed,
recovered, death, actives) in the world.

By looking at the table, the number of actives in world is still over 2
million cases. And I was quite surprise that the number of recovered is
almost one-third of confirmed cases. It indicates that good amount of
confirmed people cured by medicine or self-recovered. Since lots of
countries' hospitals face frontlines of crisis because of coronavirus,
the number of people get recovered is important number to see for
checking whether the virus blows over.

The time-line of each case is also helpful to understand overall
situation and changes. Even though number of confirmed have high slope
and uptrend, active number's slop winces little compare to past month or
weeks.

~

\begin{longtable}[]{@{}lrrrr@{}}
\toprule
update\_date & confirmed & recovered & deaths & actives\tabularnewline
\midrule
\endhead
2020-05-03 & 3506729 & 1125236 & 247470 & 2134023\tabularnewline
\bottomrule
\end{longtable}

\begin{center}\includegraphics{final.project_files/figure-latex/unnamed-chunk-9-1} \end{center}

\hypertarget{world-map-by-active-cases-number-and-rate-by-population}{%
\subsection{World map by active cases number and rate (by
population)}\label{world-map-by-active-cases-number-and-rate-by-population}}

Since we have worldwide active case, it is good to visualize what
continent/country have more active case than other. Therefore, I tried
to visualize the map in two ways because the actives case is depending
on the population of country. In the bottom two tables, the top active
countries would not show up in top active rate countries since their
number is relatively small in population.

~

\hypertarget{top-3-country-by-active-cases}{%
\subsubsection{Top 3 country by active
cases}\label{top-3-country-by-active-cases}}

\begin{longtable}[]{@{}lrrr@{}}
\toprule
region & actives & ratio\_active & population\tabularnewline
\midrule
\endhead
united states of america & 910206 & 0.2766038 & 329064917\tabularnewline
united kingdom & 158421 & 0.2345929 & 67530172\tabularnewline
russia & 116768 & 0.0800481 & 145872256\tabularnewline
\bottomrule
\end{longtable}

\hypertarget{top-3-country-by-active-rate}{%
\subsubsection{Top 3 country by active
rate}\label{top-3-country-by-active-rate}}

\begin{longtable}[]{@{}lrrr@{}}
\toprule
region & actives & ratio\_active & population\tabularnewline
\midrule
\endhead
san marino & 455 & 1.3437685 & 33860\tabularnewline
qatar & 13875 & 0.4899248 & 2832067\tabularnewline
singapore & 16779 & 0.2890769 & 5804337\tabularnewline
\bottomrule
\end{longtable}

~ The bottom two graphs shows some countries have lighter or darker
color in active rate map compare to active number itself. Norway,
Ireland, Gabon, and Chile have darker color in rate map.

~

\begin{center}\includegraphics{final.project_files/figure-latex/unnamed-chunk-12-1} \end{center}

\hypertarget{confirmed-cases-and-confirmed-ratio-by-population}{%
\subsection{Confirmed cases and Confirmed ratio by
population}\label{confirmed-cases-and-confirmed-ratio-by-population}}

To look the number deeply, I first plotted the number of confirmed cases
and confirmed rate by time in specific countries. I used rate by
population because it makes easy to compare various countries in a one
plot. Also, I chose only top 10 GDP ranked countries' cases since most
of people have interested to see the number in developed countries, and
how they deal with this situation.

Before March, most countries have very low confirmed cases, except
China. So, I checked up the first date of testing case in countries.
Even though some countries started testing in January and February, they
did not get confirmed case that much. However, after mid-March the graph
starts to have sharp increasement especially in United States.

However, rapid increase in U.S. confirmed cases does not mean that U.S.
people tend to have positive result in testing than other countries.
Therefore, if I look the right plot (confirmed rate), actually Italy has
higher confirmed rate than U.S. Also, in rate plot shows better to
understand that actually most of countries have more rapid change around
mid-March.

~

\includegraphics{final.project_files/figure-latex/unnamed-chunk-13-1.pdf}

\begin{longtable}[]{@{}ll@{}}
\toprule
region & first\_testing\tabularnewline
\midrule
\endhead
united states of america & 2020-01-18\tabularnewline
japan & 2020-02-18\tabularnewline
germany & 2020-03-08\tabularnewline
united kingdom & 2020-04-06\tabularnewline
france & 2020-02-24\tabularnewline
india & 2020-03-13\tabularnewline
italy & 2020-02-24\tabularnewline
canada & 2020-03-18\tabularnewline
\bottomrule
\end{longtable}

\hypertarget{testing-number-changes}{%
\subsection{Testing number changes}\label{testing-number-changes}}

After I figured out the confirmed case is depending on population, I
decided to compare cumulative testing number change also. Since lots of
countries do not have enough number of covid-19 testing kit, there might
be some countries have lower confirmed cases because they did not test
enough. Also, some people said the number of testing in developed
countries busted their bubble. They thought developed countries can deal
with the virus, but in fact, they also facing difficulty with supplying
sanitized product such as mask, sanitizer, gloves, and etc.

By looking at the change in testing in time, most of countries have
rapid increase stat at March, but interestingly in India and Japan has
the increase point at April even though they are Asian countries which
located close to China.

~

\begin{center}\includegraphics{final.project_files/figure-latex/unnamed-chunk-14-1} \end{center}

Note: Since China and Brazil does not provide their number of testing,
the testing cases change does not contain those two countries.

~ By looking at the upper plot, it shows that top rank countries do not
guarantee they test covid-19. Japan is rank \#2 country, but they are
the least testing country in May.

To investigate this graph further, I plot the bar plot of tested rate by
their populations. If the rate is way small, then it may indicate that
country only test a person who has serious symptom because they have
lack of physician or doctor, or their population is comparatively larger
than others.

U.S., Japan, India have relatively small test rate compare to other top
countries. Since I order the bar plot by the world rank, it clearly
shows that world ranking is not following the trend of tested ratio.

Note: Since some of the countries do not share the cumulative testing
number, I plotted top 10 countries who share it. ~

\begin{center}\includegraphics{final.project_files/figure-latex/unnamed-chunk-15-1} \end{center}

\hypertarget{active-cases-and-active-ratio-by-population}{%
\subsection{Active cases and Active ratio by
population}\label{active-cases-and-active-ratio-by-population}}

After investigating confirmed and tested number, finally I plot the
time-line of active case in the world.

By looking at the bottom graph, some of countries having downtrend after
mid-April. However, U.K., U.S., Canada, and Brazil still have uptrend
for active cases.

\includegraphics{final.project_files/figure-latex/unnamed-chunk-16-1.pdf}

\hypertarget{distribution-of-active-recover-and-death-in-confirmed-cases}{%
\subsection{Distribution of active, recover, and death in confirmed
cases}\label{distribution-of-active-recover-and-death-in-confirmed-cases}}

After the ingestions data by time series, I wondered how much the case
occupied within confirmed cases. To see the distribution well, I grab
the top 15 countries who have most confirmed cases.

The proportion is inconsistent between countries. Therefore, it is hard
to conclude the trending of each case's distribution. However, the
interesting part is death proportion. Except Russia, Peru, Germany, it
is quite noticeable. In the next step, I looked into this death portion.

~

\begin{center}\includegraphics{final.project_files/figure-latex/unnamed-chunk-17-1} \end{center}

~

\hypertarget{data-set-2}{%
\section{Data set 2:}\label{data-set-2}}

\hypertarget{distribution-of-death-cases-by-races-in-united-states}{%
\subsection{Distribution of death cases by races in United
States}\label{distribution-of-death-cases-by-races-in-united-states}}

Since the world death rate (deaths / confirmed) is around 7\%, I tried
to look into more detail information like the distribution of race.
Since U.S. is most diverse country and the confirmed number is
significantly high, I got U.S. death distribution by race information by
race from Centers for Disease Control and Prevention (CDC),
\url{https://www.cdc.gov/nchs/nvss/vsrr/covid_weekly/index.htm}.

From the data, I extract the cumulative United State death distribution
with 2 cases -- Distribution of COVID deaths and Weighted distribution
of population.

~

\begin{verbatim}
## [1] "Update date: 04/28/2020"
\end{verbatim}

\begin{longtable}[]{@{}lrrrrrr@{}}
\toprule
\begin{minipage}[b]{0.24\columnwidth}\raggedright
Indicator\strut
\end{minipage} & \begin{minipage}[b]{0.04\columnwidth}\raggedleft
White\strut
\end{minipage} & \begin{minipage}[b]{0.15\columnwidth}\raggedleft
Black or African American\strut
\end{minipage} & \begin{minipage}[b]{0.20\columnwidth}\raggedleft
American Indian or Alaska Native\strut
\end{minipage} & \begin{minipage}[b]{0.04\columnwidth}\raggedleft
Asian\strut
\end{minipage} & \begin{minipage}[b]{0.11\columnwidth}\raggedleft
Hispanic or Latino\strut
\end{minipage} & \begin{minipage}[b]{0.04\columnwidth}\raggedleft
Other\strut
\end{minipage}\tabularnewline
\midrule
\endhead
\begin{minipage}[t]{0.24\columnwidth}\raggedright
Distribution of COVID deaths (\%)\strut
\end{minipage} & \begin{minipage}[t]{0.04\columnwidth}\raggedleft
52.1\strut
\end{minipage} & \begin{minipage}[t]{0.15\columnwidth}\raggedleft
21.2\strut
\end{minipage} & \begin{minipage}[t]{0.20\columnwidth}\raggedleft
0.3\strut
\end{minipage} & \begin{minipage}[t]{0.04\columnwidth}\raggedleft
6.1\strut
\end{minipage} & \begin{minipage}[t]{0.11\columnwidth}\raggedleft
16.5\strut
\end{minipage} & \begin{minipage}[t]{0.04\columnwidth}\raggedleft
3.8\strut
\end{minipage}\tabularnewline
\begin{minipage}[t]{0.24\columnwidth}\raggedright
Weighted distribution of population (\%)\strut
\end{minipage} & \begin{minipage}[t]{0.04\columnwidth}\raggedleft
40.4\strut
\end{minipage} & \begin{minipage}[t]{0.15\columnwidth}\raggedleft
18.4\strut
\end{minipage} & \begin{minipage}[t]{0.20\columnwidth}\raggedleft
0.2\strut
\end{minipage} & \begin{minipage}[t]{0.04\columnwidth}\raggedleft
12.1\strut
\end{minipage} & \begin{minipage}[t]{0.11\columnwidth}\raggedleft
26.9\strut
\end{minipage} & \begin{minipage}[t]{0.04\columnwidth}\raggedleft
1.9\strut
\end{minipage}\tabularnewline
\bottomrule
\end{longtable}

\begin{center}\includegraphics{final.project_files/figure-latex/unnamed-chunk-19-1} \end{center}

~

After looking at the distribution of death in U.S., there were
significantly large percentage of death report from White than other
races even in weighted distribution. By looking at this, obviously we
cannot conclude a specific race tend to dead more/less from coronavirus.
But good resource to see the trend in diverse country.

\hypertarget{conclusion}{%
\subsection{Conclusion:}\label{conclusion}}

I faced a difficulty to grab right data for testing number. Since some
of the countries does not share their testing number, it was hard to
show the relationship between testing and confirm or other cases. Even
some of them count the number from 2 different sources and they do not
match. Also, some of the countries does not update the testing from
their source frequently.

However, after visualize the results, I found out higher ranked
countries does not guarantee they can handle the situation better than
other countries. Some of lower rank countries did more testing and got
more recovered cases. We can do further research about relapse cases if
we have the data.

Also, by the active graph, even though there were some countries have
down trend after April, we should keep eyes on the trend because there
is a possibility of having second wave of coronavirus.

\end{document}
